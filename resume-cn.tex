\documentclass[12pt,]{article}
\usepackage{lmodern}
\usepackage{amssymb,amsmath}
\usepackage{ifxetex,ifluatex}
\usepackage{fixltx2e} % provides \textsubscript
\ifnum 0\ifxetex 1\fi\ifluatex 1\fi=0 % if pdftex
  \usepackage[T1]{fontenc}
  \usepackage[utf8]{inputenc}
\else % if luatex or xelatex
  \ifxetex
    \usepackage{mathspec}
    \usepackage{xltxtra,xunicode}
  \else
    \usepackage{fontspec}
  \fi
  \defaultfontfeatures{Mapping=tex-text,Scale=MatchLowercase}
  \newcommand{\euro}{€}
\fi
% use upquote if available, for straight quotes in verbatim environments
\IfFileExists{upquote.sty}{\usepackage{upquote}}{}
% use microtype if available
\IfFileExists{microtype.sty}{%
\usepackage{microtype}
\UseMicrotypeSet[protrusion]{basicmath} % disable protrusion for tt fonts
}{}
\ifxetex
  \usepackage[setpagesize=false, % page size defined by xetex
              unicode=false, % unicode breaks when used with xetex
              xetex]{hyperref}
\else
  \usepackage[unicode=true]{hyperref}
\fi
\hypersetup{breaklinks=true,
            bookmarks=true,
            pdfauthor={},
            pdftitle={},
            colorlinks=true,
            citecolor=blue,
            urlcolor=blue,
            linkcolor=magenta,
            pdfborder={0 0 0}}
\urlstyle{same}  % don't use monospace font for urls
\setlength{\parindent}{0pt}
\setlength{\parskip}{6pt plus 2pt minus 1pt}
\setlength{\emergencystretch}{3em}  % prevent overfull lines
\providecommand{\tightlist}{%
  \setlength{\itemsep}{0pt}\setlength{\parskip}{0pt}}
\setcounter{secnumdepth}{0}

\date{}
\usepackage[top=1in, bottom=1in, left=1.1in, right=1.1in]{geometry}
\usepackage{tgpagella}
\usepackage[dvipsnames]{xcolor}
\ifxetex
\usepackage{xeCJK}
\setCJKmainfont[
    BoldFont={HYQiHeiX2-GEW}, 
    % BoldFont={FZLTZHK--GBK1-0}, 
    ItalicFont={STKaiti},
    % BoldItalicFont={HYQiHei-EES}
    BoldItalicFont={HiraginoSansGB-W3}
    % BoldItalicFont={FZLTXHK--GBK1-0}
% ]{STSongti-SC-Regular}
]{HYa6gj}
  %\setmainfont[Mapping=tex-text]{TeX Gyre Pagella}
\renewcommand{\baselinestretch}{1.5} % Chinsese should have larger line height
\fi

\setmainfont[
    BoldFont={MinionPro-Bold}, 
    ItalicFont={MinionPro-It}
]{MinionPro-Regular}
\usepackage{titlesec}
\titleformat{\section}{\Huge\bfseries\itshape}{\thesection}{1em}{}
\titlespacing{\section}{0pt}{-2.5em}{2.3em}
% \titleformat{\subsection}{\huge\bfseries}{\thesubsection}{2em}{\textcolor{NavyBlue}{\titlerule[0.1em]\vspace{-0.5em}\\}}
\titleformat{\subsection}{\huge\bfseries}{\thesubsection}{2em}{}
\titleformat{\subsubsection}
    {}{\thesubsubsection}{1em}{}
\titlespacing{\subsubsection}{0pt}{0pt}{-2.278em}

%\usepackage{multicol}
\pagestyle{empty}
\hyphenation{Media-Wiki}
\setlength{\leftskip}{9em}
\setlength{\hangindent}{2em}

\usepackage{enumitem}
\setlist[1]{leftmargin=8.53em, label=, itemindent=-1em}
\setenumerate[1]{leftmargin=11.25em, label=\tiny$\blacktriangleright$}

% from http://tex.stackexchange.com/a/29796/16139
\newsavebox{\zerobox}
\newenvironment{nospace}
  {\par\edef\theprevdepth{\the\prevdepth}\nointerlineskip
   \setbox\zerobox=\vtop to 0pt\bgroup
   \hrule height0pt\kern\dimexpr\baselineskip-\topskip\relax
  }
  {\par\vss\egroup\ht\zerobox=0pt \wd\zerobox=0pt \dp\zerobox=0pt
   \box\zerobox}

\setlength{\quotewidth}{0.65\textwidth}

% Redefines (sub)paragraphs to behave more like sections
\ifx\paragraph\undefined\else
\let\oldparagraph\paragraph
\renewcommand{\paragraph}[1]{\oldparagraph{#1}\mbox{}}
\fi
\ifx\subparagraph\undefined\else
\let\oldsubparagraph\subparagraph
\renewcommand{\subparagraph}[1]{\oldsubparagraph{#1}\mbox{}}
\fi

\begin{document}

\begin{nospace}\begin{flushright}
\vspace{-2em}湖南省长沙市\\
德雅路109号,410073\\
+86 150 7480 7123\\
anran.yang.china@gmail.com
\end{flushright}\end{nospace}

\section{杨岸然}\label{ux6768ux5cb8ux7136}

\begin{quote}
1988年7月生人,擅沟通,乐于学习;具有全栈开发能力,精通数种编程语言与技术框架,可独立完成从建模到架构,从后端服务设计开发到前端人机界面设计开发等工作;在地理信息领域有深入理解,尤擅时空数据分析。
\end{quote}

\subsection{教育经历}\label{ux6559ux80b2ux7ecfux5386}

\subsubsection{2010 -}\label{section}

国防科学技术大学\\
\emph{工学博士在读(硕转博),地理信息系统专业}

\subsubsection{2014 - 2015}\label{section-1}

海德堡大学,德国\\
\emph{GIScience Research Group,公派学术访问一年}

\subsubsection{2006 - 2010}\label{section-2}

北京大学\\
\emph{理学学士,地理信息系统专业,GPA 3.5/4}

\subsection{技能}\label{ux6280ux80fd}

\subsubsection{计算机}\label{ux8ba1ux7b97ux673a}

\begin{itemize}
\tightlist
\item
  网络应用程序开发(JavaScript/CSS/HTML)
\item
  地理信息服务开发(Python/C++)
\item
  高性能服务器配置管理(Linux/Docker/Bash)
\item
  软件自动化验证与测试(Tsung/PAT/Cucumber)
\item
  时空数据分析(R/C++/C/SQL)
\end{itemize}

\subsubsection{学术研究}\label{ux5b66ux672fux7814ux7a76}

\begin{itemize}
\tightlist
\item
  长尾数据统计分析
\item
  通信顺序进程(CSP)
\end{itemize}

\subsubsection{其它}\label{ux5176ux5b83}

\begin{itemize}
\tightlist
\item
  平面设计,绘画
\end{itemize}

\subsection{项目经历}\label{ux9879ux76eeux7ecfux5386}

\subsubsection{2014 -}\label{section-3}

时空数据分析 - 基于志愿者地理信息(VGI)的历史贡献数据

\begin{enumerate}
\tightlist
\item
  开放街图历史数据分析时空模型与工具集
\item
  数据质量与社群构成分析
\end{enumerate}

\subsubsection{2011 -}\label{section-4}

高性能地理信息系统

\begin{enumerate}
\tightlist
\item
  系统架构 \emph{两名主要设计者之一}
\item
  高性能地理计算流程化调度引擎 \emph{负责人}
\item
  交互设计及部分前端界面实现 \emph{负责人}
\item
  绘制引擎与数据引擎 \emph{选型与初版实现}
\item
  自动化部署工具 \emph{负责人}
\item
  自动化负载与性能测试 \emph{负责人}
\end{enumerate}

\subsection{发表论文}\label{ux53d1ux8868ux8bbaux6587}

\begin{itemize}
\tightlist
\item
  A. Yang, H. Fan, N. Jing, Y. Sun, and A. Zipf, ``Temporal Analysis on
  Contribution Inequality in OpenStreetMap: A Comparative Study for Four
  Countries'', ISPRS International Journal of Geo-Information, vol.~5,
  no. 1, p.~5, Jan. 2016.
\item
  A. Yang, H. Fan, and N. Jing, ``Amateur or professional: Assessing the
  expertise of major contributors in OpenStreetMap based on contributing
  behaviors'', ISPRS International Journal of Geo-Information, vol.~5,
  no. 2, p.~21, 2016.
\item
  A. Yang, L. Liu, L. Chen, and N. Jing, ``Geographical Workflow System
  over HPC Clusters Based on MPI'', in Proceedings of the 12th
  International Conference on GeoComputation, Wuhan, P.R.China, 2013
\item
  L. Liu, A. Yang, L. Chen, W. Xiong, Q. Wu, and N. Jing, ``HiGIS-When
  GIS Meets HPC'', in Proceedings of the 12th International Conference
  on GeoComputation, Wuhan, P.R.China, 2013
\end{itemize}

\end{document}
